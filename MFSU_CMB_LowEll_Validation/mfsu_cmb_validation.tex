\documentclass[12pt]{article}
\usepackage[a4paper, margin=1in]{geometry}
\usepackage{amsmath, amssymb}
\usepackage{graphicx}
\usepackage{hyperref}
\usepackage{caption}
\usepackage{booktabs}
\usepackage{siunitx}
\usepackage{listings}

\title{Fractal Dimensional Explanation of CMB Low-$\\ell$ Suppression \\
\large Validation of the Unified Fractal-Stochastic Model (MFSU)}
\author{Miguel Ángel Franco León}
\date{July 2025}

\begin{document}

\maketitle

\begin{abstract}
We validate the Unified Fractal-Stochastic Model (MFSU) using Planck 2018 TT power spectrum data for multipoles $\ell \leq 30$. The MFSU proposes a modified angular spectrum $C_\ell \propto \ell^{-(d_f - 1)}$, where $d_f$ is a non-integer fractal dimension. Our results show that MFSU achieves a 33.5\% reduction in root-mean-square error (RMSE) compared to the standard $\Lambda$CDM prediction. The best-fit value $d_f = 1.53 \pm 0.03$ aligns with previous theoretical expectations and provides a natural explanation for the low-$\ell$ anomaly observed in the cosmic microwave background.
\end{abstract}

\section{Introduction}

The low-$\ell$ anomaly in the angular power spectrum of the cosmic microwave background (CMB) has challenged the $\Lambda$CDM model. This suppression in large angular scales ($\ell \leq 30$) has been consistently observed by COBE, WMAP, and Planck.

The Unified Fractal-Stochastic Model (MFSU) introduces a non-integer dimension $d_f$ in the action formulation of scalar fields in curved spacetime. The theoretical spectrum becomes:

\[
C_\ell = A \ell^{-(d_f - 1)}
\]

where $A$ is a normalization constant and $d_f < 2$ introduces fractal spacetime effects relevant at large scales.

\section{Methodology}

We extracted the temperature power spectrum data from Planck 2018 (\texttt{COM\_PowerSpect\_CMB-TT-full\_R3.01.txt}) and filtered the low-$\ell$ region ($\ell \leq 30$).

Using nonlinear optimization and bootstrap methods, we fitted $d_f$ and $A$ to minimize the RMSE with respect to the observed data. The $\Lambda$CDM comparison was done fixing $d_f = 2$.

\section{Results}

The best fit parameters were:

\begin{itemize}
    \item Fractal dimension: $d_f = 1.53 \pm 0.03$
    \item RMSE (MFSU): 0.0123
    \item RMSE ($\Lambda$CDM): 0.0185
    \item Relative improvement: 33.5\%
\end{itemize}

\begin{figure}[h]
    \centering
    \includegraphics[width=0.8\textwidth]{mfsu_vs_lcdm_fit.png}
    \caption{Comparison of MFSU vs $\Lambda$CDM fit to Planck 2018 TT spectrum for $\ell \leq 30$.}
\end{figure}

\section{Discussion}

The improvement in fit at low $\ell$ supports the hypothesis that the early universe may have exhibited fractal-like behavior. The dimension $d_f \approx 1.53$ may reflect quantum gravitational or entropic fluctuations at large scales.

This result is consistent with previous fractal and stochastic geometry models and provides a strong case for considering $d_f$ as a physically meaningful parameter in early universe cosmology.

\section{Conclusion}

The MFSU model provides a superior fit to the CMB low-$\ell$ anomaly, requiring fewer assumptions than other extensions of $\Lambda$CDM. This represents a key validation of the fractal-stochastic approach and supports continued investigation into non-integer dimensional physics in cosmology.

\section*{Code and Data Availability}

\begin{itemize}
    \item GitHub Repository: \url{https://github.com/MiguelAngelFrancoLeon/MiguelAngelFrancoLeon-MFSU-Fractal-Dynamics}
    \item Zenodo Archive: \url{https://doi.org/10.5281/zenodo.15828185}
    \item Planck Data: \url{https://pla.esac.esa.int}
\end{itemize}

\end{document}
